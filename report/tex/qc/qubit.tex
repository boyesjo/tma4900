\section{Quantum states}
\subsection{The qubit}
The quantum bit, the qubit, is the building block of quantum computing.
Like the classical binary digit, it can be either 0 or 1.
But being quantum, these are quantum states, $\ket{0}$ and $\ket{1}$\footnotemark{}, and the qubit can be in any superposition of these states.
\footnotetext{
    The $\ket{\cdot}$ notation is known as a ket and is used in quantum mechanics to denote a quantum state.
    It is effectively a column vector.
    The inner product may be taken with a bra, $\bra{\cdot}$, to give a scalar.
    These inner products are then denoted by $\bra{\cdot}\ket{\cdot}$.
    Similarly, outer products are well-defined and denoted by $\ketbra{\cdot}{\cdot}$.
}
This follows from the first postulate of quantum mechanics\footnotemark, which states that an isolated system is entirely described by a normalised vector in a Hilbert space.
\footnotetext{
    As they are laid out in \cite{nielsen2012}.
}
For the qubit, this is the two-dimensional space where the states $\ket{0}$ and $\ket{1}$ are basis vectors, known as the computational basis states.
Thus, the state of a qubit can be expressed as
\begin{equation}
    \ket{\psi} = \alpha \ket{0} + \beta \ket{1} = \begin{pmatrix} \alpha \\ \beta \end{pmatrix},
    \label{eq:qubit}
\end{equation}
where $\alpha, \beta \in \mathbb{C}$.
The only requirement is that the state is normalised, $\vert\alpha\vert^2 + \vert\beta\vert^2 = 1$.
Normalisation is required due to the Born rule, as the absolute square of the coefficients is the probability of measuring the qubit in the corresponding basis state.

\subsection{The Bloch sphere}
A useful tool for visualising the state of a qubit is the Bloch sphere.
First, it should be noted that states on the form \cref{eq:qubit} are not physically unique, only the relative complex phase matters.
There is a global phase which can not be observed, and so it is not physically relevant.
Taking that and the normalisation into account, the state of the qubit can be expressed as
\begin{equation}
    \ket{\psi} = \cos\left(\frac{\theta}{2}\right) \ket{0} + e^{i\phi} \sin\left(\frac{\theta}{2}\right) \ket{1},
    \label{eq:bloch}
\end{equation}
where $\theta, \phi \in \mathbb{R}$.
Interpreting $\theta$ as the polar angle and $\phi$ the azimuthal angle, the state of the qubit can be identified with a point on a sphere.
See \cref{fig:bloch}.
The state $\ket{0}$ is typically thought of as the north pole of this sphere and $\ket{1}$ the south pole.
% \Cref{fig:bloch} shows the Bloch sphere with the state of the qubit in \cref{eq:bloch}.

\begin{figure}
    \centering
    \def\svgwidth{0.65\textwidth}
    \subimport{bloch}{bloch_sphere}
    \caption{
        The Bloch sphere.
        On it, the state of a single qubit state is represented by a point.
        The state $\ket{0}$ is the north pole, and $\ket{1}$ is the south pole.
        The latitudinal angle $\theta$ determines the probability of measuring the qubit in the state $\ket{0}$, while the longitudinal angle $\phi$ corresponds to the complex phase between the two basis states.
        From \cite{wikipedia_bloch}.
    }
    \label{fig:bloch}
\end{figure}

\subsection{Mixed states and density operators}
It is not only the superpositions of states that are important in quantum computing, but also the mixed states, states that are statistical ensembles of pure states.
Pure states are those expressible as a single ket like \cref{eq:qubit}, while mixed states arise when the preparation the system is not perfectly known or when the system interacts with the environment.
For the description of mixed states, the formalism of density operators is more useful than the state vector formalism.
If there are no classical uncertainties, the state is pure, and the density operator can be expressed a single ket-bra, $\rho = \ketbra{\psi}{\psi}$.
In a mixed state, however, some classical probabilities $p_i$ are associated with the different pure states $\ket{\psi_i}$, and the state of the system is described by the density operator
\begin{equation}
    \rho = \sum_{i=1}^n p_i \ketbra{\psi_i}{\psi_i}
    \label{eq:density}
\end{equation}
where $\ket{\psi_i}$ are the states of the system, and $\bra{\psi_i}$ are the corresponding dual vectors.
Being probabilities, the $p_i$ must be non-negative and sum to one.
Given a basis and a finite Hilbert space, the density operator can be expressed as a density matrix\footnotemark{} where the diagonal elements are the probabilities of measuring the system in the corresponding basis state.
Furthermore, it is easily seen that the density operator must be positive semidefinite and Hermitian.

\footnotetext{
    Density operators and matrices are often used interchangeably in quantum computing.
    Due to the finite number of qubits, the Hilbert spaces are always finite-dimensional, and with the canonical basis, there is a canonical way of representing density operators as matrices.
}

The Pauli matrices,
\begin{equation}
    \sigma_x = \begin{pmatrix} 0 & 1 \\ 1 & 0 \end{pmatrix}, \quad
    \sigma_y = \begin{pmatrix} 0 & -i \\ i & 0 \end{pmatrix}, \quad
    \sigma_z = \begin{pmatrix} 1 & 0 \\ 0 & -1 \end{pmatrix},
    \label{eq:pauli}
\end{equation}
together with the identity matrix serve as a basis for the real vector-space of Hermitian $2\times 2$ matrices.
Since the diagonal elements of a density matrix must sum to one, the density matrix for a single qubit can be expressed as
\begin{equation}
    \rho = \frac{1}{2} \left(I + x \sigma_x + y \sigma_y + z \sigma_z\right),
\end{equation}
where $x, y, z \in \mathbb{R}$.
Being positive semidefinite, the determinant should be non-negative, and thus it can be shown that $x^2 + y^2 + z^2 \leq 1$.
This allows density operators to be interpreted as points on the Bloch sphere or indeed within it.
Notably, pure states lie on the surface, while mixed states lie within the sphere (or rather, the Bloch ball).
A pure quantum superposition of $\ket{0}$ and $\ket{1}$ with equal probabilities would have a complex phase and lie somewhere on the equator, while a statistical mixture with equal classical probabilities of being $\ket{0}$ and $\ket{1}$ would lie in its centre.

\subsection{Systems of multiple qubits}
Although the continuous nature of the qubit is indeed useful, the true power of quantum computers lie in how multiple qubits interact.
Having multiple qubits enables entanglement, which is a key feature of quantum computing.

With two qubits, there are four possible states, $\ket{00}, \ket{01}, \ket{10}, \ket{11}$.
Each of these four states have their own probability amplitude, and thus their own probability of being measured.
A two-qubit system will therefore operate with four complex numbers in the four-dimensional Hilbert space $\mathbb{C}^{4}$.

Generally, the state of multiple qubits can be expressed using the tensor product as
\begin{equation}
    \ket{\psi_1 \psi_2 \cdots  \psi_n}
    = \ket{\psi_1} \ket{\psi_2} \cdots \ket{\psi_n}
    = \ket{\psi_1} \otimes \ket{\psi_2} \otimes \cdots \otimes \ket{\psi_n}
    \label{eq:tensor}.
\end{equation}
What makes this so powerful is that the state of a multi-qubit system has the general form
\begin{equation}
    \begin{aligned}
        \ket{\psi_1 \psi_2 \cdots  \psi_n}
         & = c_1 \ket{0\dots 00} + c_2 \ket{0\dots 01} + \cdots + c_{2^n} \ket{1\dots 11} \\
         & = (c_1, c_2, \dots, c_{2^n} )^\top
        \in \mathbb{C}^{2^n},
        \label{eq:superposition}
    \end{aligned}
\end{equation}
which means that with $n$ qubits, the system can be in any superposition of the $2^n$ basis states.
Operating on several qubits then, one can do linear algebra in an exponentially large space.
This is a key part of the exponential speed-ups possible with quantum computers.