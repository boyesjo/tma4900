\section{Fixed arms}
\label{sec:sim_fixed_arms}
First, the algorithms are tested on fixed bandit instances with two arms.
The mean of the first arm is set to $0.5$ and the mean of the second arm is set to $0.505$, an instance also tested in~\autocite{wan2022}.
The results are shown in \cref{fig:big2}, agreeing with the original paper; QUCB greatly outperforms UCB.
However, Thompson sampling which was not considered in~\autocite{wan2022}, is not that far behind.

Note the jagged and periodically completely flat behaviour of the QUCB algorithm regret.
This is due to the long QMC periods in which its quantum advantage is gained; it must repeatedly pull the same arm for the QMC estimates to be produced.
When the algorithm pulls the optimal arm, the change in regret is obviously zero.
Additionally, these periods get longer as the algorithm progresses, as the QMC estimates become more accurate.
The exponential lengthening of the QMC periods assure that such non-smooth and piecewise linear behaviour is to be expected.
Because of these long periods, there is not as many possible trajectories as with the classical algorithms, so the jaggedness persists despite averaging over many simulations.

\begin{figure}
    \centering
    \newcommand{\myoptions}{
        width=10cm,
        height=8cm,
        xlabel={Turn},
        ylabel={Regret},
        legend entries={UCB, QUCB, Thompson},
        legend pos=north west,
        legend cell align=left,
        mystyle,
        largexnumbers,
    }
    \subimport{figs}{big2}
    \caption{QUCB1 regret for two arms, with mean 0.5 and 0.505.}
    \label{fig:big2}
\end{figure}

\subsection{Low and high probabilities}
Next, mean more extreme mean values of $0.01$ and $0.005$
with a reward gap of $0.005$ equal to the above, are tested.
\Cref{fig:low_prob_fix} contains the results.
Here, QUCB beats UCB thoroughly still, both algorithms achieving similar regrets as in \cref{fig:big2}, but now Thompson sampling is clearly superior.




\begin{figure}
    \centering
    \begin{subfigure}{\textwidth}
        \centering
        \newcommand{\myoptions}{
            width=10cm,
            height=8cm,
            xlabel={Turn},
            ylabel={Regret},
            legend entries={UCB, QUCB, Thompson},
            legend pos=north west,
            legend cell align=left,
            mystyle,
            largexnumbers,
        }
        \subimport{figs}{low_prob_fix}
        \caption{Two arms, means 0.01, 0.005.}
        \label{fig:low_prob_fix}
    \end{subfigure}
    \\[3ex]
    \begin{subfigure}{\textwidth}
        \centering
        \newcommand{\myoptions}{
            width=10cm,
            height=8cm,
            xlabel={Turn},
            ylabel={Regret},
            legend entries={UCB, QUCB, Thompson},
            legend pos=north west,
            legend cell align=left,
            mystyle,
            largexnumbers,
        }
        \subimport{figs}{high_prob}
        \caption{Regret for two arms, means 0.99, 0.9905.}
        \label{fig:high_prob}
    \end{subfigure}
    \caption{QUCB1 regret for especially high and low means.}
    \label{fig:low_high_prob}
\end{figure}

At these extreme values, it seems the gap must shrink for QUCB to be able to outperform Thompson sampling.
This is shown in \cref{fig:high_prob}, where the gap is reduced to $0.0005$,
Even so, the QUCB advantage is not as pronounced as in \cref{fig:big2}.



\subsection{Four arms}
As a final test, the number of arms is increased to four.
The results are plotted in \cref{fig:four_arms}.
Only cases with two arms were tested in the original paper, so this is a new test, further validating the correctness of the algorithm and its implementation.
The results are similar to the two-arm case, with QUCB outperforming UCB.
At these particular reward means, Thompson sampling performs very similarly to QUCB.
In general, QUCB provides no advantages or noteworthy changes from UCB with respect to the number of arms, so its behaviour should be expected to be similar as UCB when the number of arms is increased.

\begin{figure}
    \centering
    \newcommand{\myoptions}{
        width=10cm,
        height=8cm,
        xlabel={Turn},
        ylabel={Regret},
        legend entries={UCB, QUCB, Thompson},
        legend pos=north west,
        legend cell align=left,
        mystyle,
        largexnumbers,
    }
    \subimport{figs}{four_arms}
    \caption{Regret for four arms, with mean 0.5, 0.51, 0.52 and 0.53.}
    \label{fig:four_arms}
\end{figure}