\chapter{Simulations}
\label{sec:simulations}

While the upper bound shown in \cref{sec:wan2022} implies QUCB1 advantage, it does not necessarily mean it is supreme in all cases or even on average.
It is therefore of much interest to test it on a variety of problems and compare its average performance to other algorithms — not only with the classical UCB1 to which it is compared in the original paper, but also with the more performant Thompson sampling algorithm.

All algorithms were implemented in Python 3~\autocite{python} with quantum computing support provided by IBM's Qiskit library~\autocite{qiskit}.
The experiments were run on an external computation server with an Intel Xeon E5-2690 v4 CPU and 768 GB of RAM, permitting 28 concurrent simulations.
For fixed bandit instances, 100 simulations were run for each algorithm.

To limit the scope and computational resources required, only Bernoulli rewards were considered.
As noted in~\autocite{wan2022}, setting the confidence parameter to a higher value than the theoretically desirable $1/T$ leads to better performance.
It was thus set to $0.01$ for all experiments.
Moreover, the constant $C_1$ which was only defined existentially was arbitrarily set to $2$ across all experiments, as it was not clear how to choose it in a principled manner, and this value seemed able to reproduce the results of~\autocite{wan2022}.


\subimport{}{fixed}
\clearpage


\subimport{}{bayesian}
\clearpage

\subimport{}{rl}
