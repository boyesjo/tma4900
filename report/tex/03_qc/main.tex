\chapter{Quantum computing}
\label{chap:qc}
The field of quantum computing is split into two main branches: the development of quantum hardware and the study of algorithms that use such hardware.
Only the second branch is relevant for this thesis, and even so only a brief explanation is offered here.
For more details, see \cite{nielsen2012} for a rigorous, complete description or \cite{qiskit_textbook} for an introduction focused on programming.
% This chapter is primarily based on the Qiskit textbook \cite{qiskit_textbook}, \citetitle{schuld2021a} by \textcite{schuld2021a} and \citetitle{nielsen2012} by \textcite{nielsen2012}.
% The discussion of variational quantum algorithms in \cref{sec:vqa} is based on the review by \textcite{cerezo2021}.
Any reader should have a basic understanding of linear algebra and classical computing.
Knowledge of quantum mechanics is not assumed, albeit certainly helpful.

\subimport{}{qubit}
\subimport{}{operating}
\subimport{}{algos}
% \subimport{}{nisq}
% \subimport{}{vqa}




