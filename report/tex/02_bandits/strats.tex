\section{Strategies}
\begin{table}
    \centering
    \caption{
        Comparison of strategies.
    }
    \label{tab:strategies}
    \begin{tblr}{
            width=\linewidth,
            colspec={X[l] X[r] X[r]}
        }
        \toprule
        Strategy       & Regret      & Tuning    \\
        \midrule
        Random         & Linear      & NA        \\
        Greedy         & Linear      & NA        \\
        Epsilon-greedy & Linear      & Difficult \\
        Epsilon-decay  & Logarithmic & Difficult \\
        UCB            & Logarithmic & Barely    \\
        Thompson       & Logarithmic & Priors    \\
        \bottomrule
    \end{tblr}
\end{table}

\subsection{Explore-only}
Pure exploration is obviously a suboptimal strategy, but it is a good baseline against which to compare.
It can be implemented by selecting an arm uniformly or in order, performing poorly either way.
A random arm-selection procedure is described by \cref{alg:random}.
\begin{algorithm}
    \caption{Random arm selection}
    \label{alg:random}
    \begin{algorithmic}
        \State Sample $a$ from $\mathcal{A}$ uniformly
        \State \Return $a$
    \end{algorithmic}
\end{algorithm}

It is easy that the regret is
\begin{equation}
    R_T = T\left(\mu^* - \frac{1}{k}\sum_{i=1}^k \mu_i\right),
\end{equation}
which is necessarily linear in $T$.
This motivates the search for an algorithm with sublinear regret.

\subsection{Greedy}
Tending away from pure exploration to exploitation, a greedy algorithm will always select the arm with the highest empirical mean.
Here, all arms are pulled an initial $m \geq 1$ times, after which estimated means are used to select the best arm.
Then, the arm with the highest empirical mean is selected for all remaining turns.
The arm-selection procedure is listed in \cref{alg:greedy}, where $\hat{\mu}_a$ is the estimated mean of arm $a$.
\begin{algorithm}
    \caption{Greedy arm selection}
    \label{alg:greedy}
    \begin{algorithmic}
        \If{$t \leq mk$}
        \State \Return $(t \mod k) + 1$
        \Else
        \State \Return $\argmax_{a \in \mathcal{A}} \hat{\mu}_a$
        \EndIf
    \end{algorithmic}
\end{algorithm}

With greedy selection, the expected regret is clearly still linear in the horizon, as there is a non-zero probability of selecting the wrong arm.
Still, there is a chance of achieving zero regret and the constant factor is reduced compared to pure exploration selection.
To improve hereupon, it is necessary to occasionally explore other arms, which leads into the epsilon-greedy algorithm.

\subsection{Epsilon-greedy}
The problem with the greedy algorithm is that it may be unlucky and not discover the best arm in the initial exploration phase.
To mitigate this, the epsilon-greedy algorithm can be used.
In this algorithm, the arm that is presumed to be best is pulled with probability $1-\epsilon$, while in the other $\epsilon$ proportion of the turns, the arm is selected uniformly at random.
This ensures convergence to correct exploitation as the horizon increases, and it will generally reduce the regret.

Still, with a constant $\epsilon$, a constant proportion of the turns will be spent exploring, keeping the regret necessarily linear in the horizon.
Choosing $\epsilon$ is a trade-off between exploration and exploitation and can significantly affect the regret.

\begin{algorithm}
    \caption{Epsilon-greedy arm selection}
    \label{alg:eps_greedy}
    \begin{algorithmic}
        \If{$t \leq mk$}
        \State \Return $(t \mod k) + 1$
        \Else
        \State Sample $u$ from $[0,1)$ uniformly
        \If{$u < \epsilon$}
        \State Sample $a$ from $\mathcal{A}$ uniformly
        \State \Return $a$
        \Else
        \State \Return $\argmax_{a \in \mathcal{A}}\hat{\mu}_a$
        \EndIf
        \EndIf
    \end{algorithmic}
\end{algorithm}


\subsubsection{Epsilon-decay}
To remedy the linear term in the regret, modifications to the epsilon-greedy algorithm have been proposed wherein $\epsilon$ is a function of the current time step $t$.
Specifically, in order to achieve sublinear regret, it is necessary to decay $\epsilon$ towards zero.
Decreasing $\epsilon$ over time makes intuitive sense; exploration is more crucial in the beginning stages of the algorithm, whereas exploitation is more important when the agent has more reliable estimates of the reward means.
For example, one successful strategy is to set $\epsilon \sim 1/t$, which has been shown to achieve logarithmic instance-dependent regret \autocite{auer2002}.
It is worth noting, however, that the optimal decay rate depends on the specific instance, and achieving logarithmic regret can be challenging in practice \autocite{bubeck2012}.


\subsection{Upper confidence bounds}
The upper confidence bound (UCB) algorithm is a more sophisticated algorithm based on estimating an upper bound for the mean of each arm.
One always chooses the arm whose upper confidence bound is highest, a principle known as \enquote{optimism in the face of uncertainty}.
This should make sense, as if the wrong arm appears best, it will be pulled more often and the empirical mean will be corrected, while the true best arm with its larger bound will eventually become highest and so pulled.
When exploiting the actual best arm, the agent can trust it to be the best, as the confidence bound will remain above those of all the other arms.
In addition, by increasing the confidence bounds as the number of pulls increases, getting stuck in suboptimality is avoided.

Assuming rewards in $[0,1]$ and using Hoeffding's inequality, one has
\begin{equation}
    p
    = P(\mu_a > \hat{\mu_a} + \text{UCB}_a)
    \leq \exp(-2T_a \text{UCB}_a^2),
\end{equation}
where $\text{UCB}_a$ is the upper confidence bound for arm $a$ and $T_a$ is the number of times arm $a$ has been pulled.
Solving for $\text{UCB}_a$ gives
\begin{equation}
    \text{UCB}_a = \sqrt{\frac{-\ln p}{2T_a}}.
\end{equation}
Letting $p(t) = t^{-4}$ gives
\begin{equation}
    \text{UCB}_a = \sqrt{\frac{2 \ln t}{T_a}},
\end{equation}
which is a common choice for the upper confidence bound, leading to the UCB1-algorithm.
In \autocite{auer2002}, it is shown that UCB1 achieves
\begin{equation}
    R_T
    \leq
    \left( 8 \sum_{a \in \mathcal{A} \setminus a^*} \frac{\log T}{\Delta_a} \right)
    + \left(1 + \frac{\pi^2}{3}\right) \sum_{a \in \mathcal{A}} \Delta_a,
\end{equation}
which only misses instance-dependent optimality by some constant factor.
Nor is UCB1 truly minimax-optimal, is its instance-independent upper bound is $O(kT \log T)$ \autocite{bubeck2012}.


Regardless of the assumptions made and the bandit class, the procedure follows as in \cref{alg:ucb}.

Many variants of the algorithm exist; different assumptions about the distributions change the confidence bounds.
While the choice of $p$ is arbitrary, it is less of nuisance than the choice of $\epsilon$ in the epsilon-greedy algorithm, with specific choices of $p$, such as the UCB1-algorithm, being well-studied and known to perform well.
For example, MOSS, a modification of UCB1, has be shown to minimax-optimal for $\mathcal{E}_{[0,1]}^k$ \autocite{audibert2009}.
Further, incorporating estimates of second moments improve performance in some cases \autocite{audibert2009a}, while incorporating the whole empirical distributions of observed rewards appears be the most effective approach \autocite{maillard2011}.

\begin{algorithm}
    \caption{UCB arm selection}
    \label{alg:ucb}
    \begin{algorithmic}
        \If{$t \leq k$}
        \State \Return $t$
        \Else
        \State \Return $\argmax_{a \in \mathcal{A}} (\hat{\mu}_a + \text{UCB}_a)$
        \EndIf
    \end{algorithmic}
\end{algorithm}


\subsection{Thompson sampling}
Thompson sampling is a Bayesian approach to the multi-armed bandit problem, being the original approach to the problem \autocite{thompson1933} in 1933, though only in the case of two arms and Bernoulli rewards and without any theoretical guarantees.
This method is noteworthy for its ability to incorporate Bayesian modelling concepts into the fundamentally frequentist problem of multi-armed banditry.
The idea is to sample from the posterior distribution of the means of the arms and pull the arm with the highest sample.

It was first in 2012 that Thompson sampling was proven asymptotically optimal for Bernoulli rewards with uniform priors \autocite{kaufmann2012}.
Namely, for every $\epsilon > 0$, there exists a constant $C_\epsilon$ such that
\begin{equation}
    R_T
    \leq
    (1 + \epsilon)
    \sum_{a \in \mathcal{A} \setminus a^*}
    \Delta_a \frac{\log T - \log \log T}{D(P_a \mathrel{\Vert} P^*)}
    + C_\epsilon,
\end{equation}
where $P_a$ is the reward distribution of arm $a$ and $P^*$ is the reward distribution of the best arm.
Meanwhile, the minimax regret is can be bounded by either $O(\sqrt{kT \log T})$ \autocite{agrawal2013} or $O(\sqrt{kT \log k})$ \autocite{agrawal2017}, neither of which is quite minimax-optimal.

Also for Gaussian rewards, it was proven asymptotically optimal with uniform priors \autocite{honda2014}.
Notably, the Jeffreys prior was shown to be inadequate in achieving optimal regret, highlighting the importance of the prior selection for the algorithm's performance.

\begin{algorithm}
    \caption{Thompson sampling arm selection}
    \label{alg:thompson}
    \begin{algorithmic}
        \For {$a \in \mathcal{A}$}
        \State Sample $\theta_a \sim P(\mu_a \mid \mathcal{D})$
        \EndFor
        \State \Return $\argmax_{a \in \mathcal{A}} \theta_a$
        \State Update posterior for arm $a$ with reward $X_t$
    \end{algorithmic}
\end{algorithm}

One of the key advantages of Thompson sampling is that it can natively incorporate prior knowledge about the arms, whereas doing so with the above methods would require some sort of ad-hoc manipulation of the recorded rewards and arm pulls.
Furthermore, empirical results generally indicate better performance than UCB variants \autocite{kaufmann2012}.
Still, Thompson sampling is not without its drawbacks.
The algorithm can be computationally expensive, as it requires sampling from the posterior distribution for each arm at each time step.
Even with conjugate priors, the computational costs of sampling will be higher than the simple computations required by UCB and epsilon-greedy algorithms.

\subsection{The doubling trick}
Given an algorithm reliant on knowing the horizon $T$, it is possible to use the doubling trick to achieve similar regret regardless of the horizon, creating an anytime algorithm.
The doubling trick is a simple idea: simply first run the algorithm for $T$ steps, then $2T$ steps, $4T$ ad infinitum, possibly with some other geometric factor.
It was first introduced in \autocite{auer1995}, and has since been proven to conserve minimax regrets, but not instance-dependent regrets \autocite{besson2018}.
Using instead exponential progression, it is possible to maintain instance-dependent regret bounds instead of minimax optimality \autocite{besson2018}.
