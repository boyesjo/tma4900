\section{Quantum agents}
\label{sec:quantum_agents}

As quantum neural networks are shown to be usable function approximators, they can be used in the same way classical neural networks are used in classical deep reinforcement learning algorithms.
In this section, we will discuss how quantum neural networks can be used as agents in reinforcement learning algorithms.
For this, several designs have been proposed, though the field of quantum reinforcement learning is still much less developed than quantum supervised learning is.

Like with QML, the two fields of reinforcement learning and quantum machine learning can be combined in several ways.
The entire environment or Markov decision process can be assumed to be quantum, such as is done in~\autocite{ying2021}.
There, approaches based on dynamic programming are used to solve the problem.
Nonetheless, continuing the theme of solving classical problems with quantum methods, this thesis will focus mainly on the use of quantum neural networks as agents in classical reinforcement learning algorithms.
Still, as will be seen, quantum agents provide the greatest benefits when the interacting with the environment is quantum also.

In~\autocite{chen2020}, quantum neural networks are used to approximate Q-functions, which are used in reinforcement learning.
Moreover, by including experience replay and a target network, the model proposed is effectively a true translation of the classical DQN algorithm to the quantum setting, a QDQN.
The authors note that the model is more efficient than classical DQN in terms of memory usage and parameter counts.

Quantum Q-learning is also studied in~\autocite{skolik2022}, where a QNN-based Q-learning is tested in several standard benchmark environments, including the cart-pole environment.
The model achieves results comparable to those of a DQN.
As is the case with classical reinforcement learning, it appears that hyperparameter tuning and model architectures matter more than the pure parameter count for the performance of the model.

The authors of~\autocite{jerbi2021} propose to use a QNN to approximate the policy function in reinforcement learning and train it with the REINFORCE algorithm.
Their model is shown to solve several basic benchmark problems, including the cart-pole environment, with performance comparable to that of classical neural networks.
Furthermore, by designing quantum environments particularly suited for the quantum model, it is shown to outperform classical neural networks.

By letting the agent communicate over a quantum channel,~\autocite{saggio2021} show that the learning of an agent can be accelerated.
Somewhat similarly, in~\autocite{hamann2022}, hybrid agents are shown to learn quadratically faster than purely classical agents.