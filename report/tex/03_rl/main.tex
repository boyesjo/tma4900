\chapter{Reinforcement learning}
\label{chap:rl}

Machine learning lies at the intersection of statistics, computer science and optimisation.
The central idea is to design an algorithm that uses data to solve a problem, and in so avoid explicitly programming a solution.
With ever more data available and with ever more powerful computers, machine learning has become a powerful tool in many fields, solving problems previously thought intractable.
Such algorithms or models can be used for a plethora of tasks, which are mainly divided into three main categories:
\begin{description}

    \item[Supervised learning]
        Given data with corresponding labels, find the relationship and try to assign correct labels to new, unseen data.

    \item[Unsupervised learning]
        Given data, find some underlying structure, patterns, properties or relationships, such as clusters or outliers.

    \item[Reinforcement learning]
        Given an environment with a set of possible actions, such as a game, explore different strategies and determine one that optimises some reward.

\end{description}

This chapter will focus on the third category, reinforcement learning (RL).
While bandits are a useful tool for the study of interaction of environments, they are limited in their ability to model complex environments wherein actions have long-term consequences.
The following will serve mostly as an introduction to the topic of RL, and will not go into the mathematical details of the state-of-the-art algorithms.

Unlike bandit problems which can be considered analytically and solved somewhat optimally, MDPs are generally intractable and far too complex to be solved optimally.
Exempli gratia, the game of go has approximately $10^{170}$ possible states, far too complicated to be solved exactly optimally.
However, with modern machine learning techniques, namely neural networks, it is possible to learn good policies and beat the best human players, which was famously demonstrated with AlphaGo in 2016~\autocite{silver2016}.

In contrast to bandits whose performance is important from the start, reinforcement learning is generally tackled by first training the model on a set of data or a simulated environment before deploying it to the real problem at hand.
Consequently, the exploration-exploitation dilemma is not as pressing in reinforcement learning as it is in bandits, but it is still present.
While learning strategies, there is indeed a trade-off between optimising the more promising strategies contra attempting something radically different.
Epsilon-greedy strategies can be used, though more advanced strategies with adaptive exploration rates appear to be more effective~\autocite{tokic2011}.

The first section of this chapter will introduce the Markov decision process (MDP), which is the mathematical model of reinforcement learning, and it will be based primarily on~\autocite{lattimore2020,sutton2018}.
After that, neural networks will be introduced as a tool for solving machine learning problems, and the chapter will conclude with a brief overview of the state-of-the-art algorithms for reinforcement learning.


\subimport{}{mdp}
\subimport{}{nn}
\subimport{}{algs}